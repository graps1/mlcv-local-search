\section{Notation}
Let $A$ be some set. We will denote the set $\{ \{ a_1, \dots, a_k \} \subseteq A : a_1,\dots,a_k\ \text{pairwise distinct}\}$, which contains all $k$-ary subsets of $A$, by $\binom{A}{k}$. For the same set $A$, $P_A$ denotes the set of all partitions of $A$. For a partition $\Pi \in P_A$ and an element $a \in A$, $[a]_\Pi$ denotes the set in $\Pi$ that contains $a$, of which there is exactly one. If $A$ is finite, $\Uc(A)$ denotes the uniform distribution over $A$, i.e. the distribution that assigns every element in $A$ the same probability (namely: $1/|A|$). If we want to indicate that a random variable ${\bf X}$ has probability distribution $\mathcal{Q}$, we will write ${\bf X} \sim \mathcal{Q}$. If $f$ is a function from the domain of a random variable ${\bf X} \sim \mathcal{Q}$ to the real numbers, its expected value is written as $\E_{{\bf X} \sim \mathcal{Q}}\left[f({\bf X})\right]$ and defined as $\E_{{\bf X} \sim \mathcal{Q}} \left[f({\bf X})\right] = \sum_{i=1}^n \mathcal{Q}(x_i) f(x_i)$, if the domain of ${\bf X}$ is finite and given by $\{x_1,\dots,x_n\}$ (we will only consider finite domains).

\section{Introduction}
Let $V = \{ v_1,\dots,v_n \}$ be a finite set of vertices. Associated with this set are functions $c,c': \binom{V}{3} \mapsto \R$ which define cost-structures on $3$-ary subsets of $V$. For a $3$-ary subset $\{u,v,w\} \in \binom{V}{3}$ and a given partition $\Pi \in P_V$, we define the cost of $\{u,v,w\}$ with respect to $\Pi$ as  
\begin{align*}
    \ell(\{u,v,w\},\Pi) = \begin{cases}
        c(\{u,v,w\}) & \text{if}\ [u]_\Pi \neq [v]_\Pi, [u]_\Pi \neq [w]_\Pi, [w]_\Pi \neq [v]_\Pi \\
        c'(\{u,v,w\}) & \text{if}\ [u]_\Pi = [v]_\Pi = [w]_\Pi \\
        0 & \text{otherwise.}
    \end{cases}
\end{align*}
This can be interpreted as follows: whenever $u,v$ and $w$ are part of pairwise different sets in $\Pi$, the cost of $\{u,v,w\}$ is equal to the costs as defined by $c$. If they are part of the same set, the cost of $\{u,v,w\}$ is equal to the costs as defined by $c'$. Otherwise, if neither of the above is the case, the cost of $\{u,v,w\}$ is just $0$. Based on this definition, we are confronted with problems of the form
\begin{align*}
    \Pi^* = \argmin_{\Pi \in P_V} \sum\nolimits_{\{u,v,w\} \in \binom{V}{3}} \ell(\{u,v,w\},\Pi)
\end{align*}
i.e. we wish to find a partition $\Pi^*$ of $V$ that minimizes some objective function over $3$-ary subsets of $V$. This problem is hard in many respects: one, the amount of possible partitions grows large very fast (with growing $n$). Two, even computing the objective function for a given partition is prohibitive, since summing over all $3$-ary subsets takes almost $n^3$ steps (and to add to that, not even $n^2$ steps are feasible if $n$ becomes larger). Thus, we are interested in a way of approximately solving this problem by the use of a local search algorithm and some considerations on the objective function.
